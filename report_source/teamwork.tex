\subsection{Team organization}

In order to have an overview of the tasks, we are using the software Trello.
With this, we can write all tasks that we need to do, who is responsible of
the task and the progression of the project: we can see what is already done
and what the other members are currently working on. We communicate with each
other using Slack in which we put information and additional contents related
to the project. \\
Using Trello was really helpful as it always gave us the feeling that we were
progressing, even when we didn't get substantial improvements, we knew that we
had worked and that we were closer to our goal. This helped us stay motivated
at all times.\\

After knowing what tasks we had to do, we split the work based on preferences of
everyone. 
Quentin was chosen team leader, he worked on a lot of things in the project and 
supervised the other team members. He worked on the computation of the affinity 
graph from the output of the neural networks. When working on the second paper,
the improved MALIS, he implemented the new loss and optimized it. He did data
augmentation, worked on the U-net with Annie and Raphaël and on image
visualization.
Annie worked on the datasets, both by exploring them and preparing them. 
Josselin worked on the computation of the loss and the maximum spanning tree. 
He then tried to generates a 6 adjacency graph for 3D images with Higra, using 
C++ and made an extension. After that, he focused on the post processing,
especially Mumford-Shah method.
Tiphanie worked on the evaluation and the image generation from the output of
our neural network. She then studied the different post-processing we could apply
to our outputs to get better results and tested them.
Raphaël, who joined the team for the second semester, first immersed himself in the
project subject and what we have done so far to have a better overall understanding
and what needs to be implemented in the second semester. He then worked on the 
implementation of the U-net and tweaked it with Annie and Quentin. He also created
the video explaining our subject and what we have done in the project.
We kept everybody informed at all time of what other people did so that
everybody could understand every part of the work.\\

Every week, we had a meeting with our supervisor in which we discussed about
what we did during the week, our issues and solutions if we had some and what
we will do afterwards. During each meeting, a different person lead the
discussion. This helped us improve our communication skills and understanding
of the project as we had to present both our work and the works others did, 
and made sure that everybody spoke for the same amount of time overall. 
We implemented this since
discussions were almost one-sided before which was not the best solution.\\
Since the beginning of the quarantine, we are doing our meetings with Hangouts,
allowing us to talk to each other in real time. We also use more social networks
like Discord, allowing us to talk with the team members when working. We can also
show our screen to the others. \\

We write a report every week in which we note what we did in
the week to prepare for the meeting and to show results, or more graphical
elements. It also allows us to keep track of our progress during the project.
We wrote all the codes in documented Jupyter notebooks that we will clean and put on GitHub. \\

\subsection{Task distribution}

\begin{longtable}{ |p{0.15\linewidth}|p{0.15\linewidth}|p{0.15\linewidth}|p{0.15\linewidth}|p{0.15\linewidth}| } 
	\hline
	Annie & Josselin & Quentin & Tiphanie & Raphaël \endhead
	\hline
	Find dataset & Computation of MST & Finalize the saliency notebook & Affinity graph thresholding & Learn PyTorch \\ 
	\hline
	Exploration of dataset & Computation and optimisation of the path between i and j & Add features in the saliency notebook & Computation of connected components &
	Implement U-net\\ 
	\hline
	Analyze what is the intput and output of the neural network and their size & Computation of the maximum path in the MST & Graph generation from output of neural network & Image generation & Tweak U-net hyperparameters\\ 
	\hline
	Understand how to use hdf files & Computation of the loss & Preparation of the dataset & Finishing inference by creating segmentation & Creation of the video with Manim\\
	\hline
	Create architecture of convolutional neural network & Find interesting patches in dataset & Get vertices pairs in same and different object & Learn PyTorch &\\
	\hline
	Load ISBI-2012 data & Train and evaluate on ISBI in 2D & Train the network on maximin affinty & Study post processing & \\
	\hline
	Learn PyTorch & Learn PyTorch & Image reconstruction with inference & Test watershed hierarchy functions & \\
	\hline
	Implement U-net & Create 6 connected graph and an add it on Higra & Fiji for evaluation on ISBI & Test other methods: region adjacency graph, horizontal cut &\\
	  \hline
	Study contrast problems between images & Work on Mumford-Shah post processing & 
	Load ISBI-2012 data & &\\
	  \hline
	Image normalization & & Train and evaluate on ISBI in 2D & &\\
	  \hline
	Prepare BSDS dataset & & Learn PyTorch & &\\
	  \hline
	Prepare notebooks for training data & & Implement two passes loss (MALIS constrained loss) and optimize it & &\\
	  \hline
	Tweak U-net hyperparameters & & Incorporate new loss into the U-net and tweak its hyperparameters & &\\
	  \hline
	  & & Pre-processing on CREMI dataset & &\\
	  \hline
	\caption{List of the main tasks and their repartition (from our Trello board)}
	\label{tab:tasks}
\end{longtable}

As we can see in table~\ref{tab:tasks} we tried to distribute the work based on
affinity, and even though the table is unbalanced, everybody put their fair
share of work into the project.\\

One aspect that is absent from this table is all that we learned.
As we will detail in the next section, everybody did not have the same knowledge
at the beginning of the project and it took everybody a different amount of
time to learn the necessary information.\\

\clearpage
\subsection{Obstacles and overcoming them}

The subject of this project was quite difficult at first because we did not have
any knowledge on image segmentation or morphology. Moreover, it was also hard to read and
understand the scientific papers because they were written in English and
we did not have the knowledge to understand them clearly at first. To overcome this difficulty, we planned several
hours to go through the important things that we needed to retain, and over
time we were able to grasp more detail from the papers since our knowledge
improved throughout the project.\\

After knowing what we had to do, the next step was to figure out how to implement
these ideas.\\
We took some time to know exactly what we needed to do and how to do it in details for the implementation.
Once we knew, we just needed to find the right functions in Higra.
It took us some time to understand how the functions work, as we never used them before.
Documentation is available online and commented notebooks showing examples 
are also provided, helping us a lot to understand their implementations. 
We spent quite a lot of time figuring out how to use them at the beginning, 
but we managed to familiarize ourselves with the library as time passed.\\

Afterwards, we choose to use the PyTorch library for the deep learning
component of the project, which nobody was familiar with in the group. 
This lack of experience with it meant that we had to spend more time learning
it. However there is a nice tutorial on PyTorch's website that allowed us to
be able to use it fairly quickly. Their documentation on Autograd was also
really helpful when debugging the gradient history loss. Overall the PyTorch
API was really helpful and allowed us to use PyTorch with a relative ease.\\

The one main big difference that we had compared to the first semester was 
the quarantine. Because of that, we could not meet with each other and work 
in a proper environment. First because of the internet connection, we also 
may not have good computers with enough RAM to run some codes on Jupyter Notebook. 
That aside, we adapted to the situation by continuing to get in touch with each 
other, by using social networks like I said earlier. We planned sessions 
during the week to work on the project and kept doing written reports and meetings.\\

During this project, we received a lot of help from Quentin. This project would
have been really hard to do if he was not there, having no great knowledge in
image segmentation and machine learning. Thanks to him, we were able to have a
better understanding of the papers by having him explain them to us. It would have taken us
a lot longer to understand the papers and how to implement them without him and we would 
not be able to present decent results in the end of the first semester presentation.\\

Overall, being aware of the work done by others helped us immensely when
facing issues as we could all think together and challenge each other's
understanding of different topics. This allowed us to overcome all those
difficulties more easily and in a more enjoyable way. 

