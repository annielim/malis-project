\subsection{Team organization}

In order to have an overview of the tasks, we are using the software Trello.
With this, we can write all tasks that we need to do, who is responsible of
the task and the progression of the project: we can see what is already done
and what the other members are currently working on. We communicate with each
other using Slack in which we put information and additional contents related
to the project. \\
Using Trello was really helpful as it always gave us the feeling that we were
progressing, even when we didn't get substantial improvements, we knew that we
had worked and that we were closer to our goal. This helped us stay motivated
at all times.\\

After knowing what tasks we had to do, we split the work based on preferences of
everyone. Quentin was chosen team leader, he worked on a lot of things in the
project and supervised the other team members. Annie worked on the datasets,
both by exploring them and preparing them.
Josselin worked on the computation of the loss and the maximum spanning tree.
Tiphanie worked on the evaluation and the image generation from the output of
our neural network.
We kept everybody informed at all time of what other people did so that
everybody could understand every part of the work.\\

Every week, we had a meeting with our supervisor in which we discussed about
what we did during the week, our issues and solutions if we had some and what
we will do afterwards. During each meeting, a different person lead the
discussion. This helped us improve our communication skills and understanding
of the project as we had to
present both our work and the works others did, and made sure that everybody
spoke for the same amount of time overall. We implemented this since
discussions were almost one-sided before which was not the best solution.\\

We also wrote a report every week in which we wrote what we did in
the week to prepare for the meeting and to show results, or more graphical
elements. We wrote all the codes in documented jupyter notebooks that we will clean and put on GitHub. \\

\subsection{Task distribution}

\begin{table}[!hbtp]
	\centering
	\begin{tabular}{ |p{0.2\linewidth}|p{0.2\linewidth}|p{0.2\linewidth}|p{0.2\linewidth}| } 
		\hline
		Annie & Josselin & Quentin & Tiphanie \\
		\hline
		Find dataset & Computation of MST & Finalize the saliency notebook & Affinity graph thresholding \\ 
		\hline
		Exploration of dataset & Computation and optimisation of the path between i and j & Add features in the saliency notebook & Computation of connected components \\ 
		\hline
		Analyze what is the intput and output of the neural network and their size & Computation of the maximum path in the MST & Graph generation from output of neural network & Image generation \\ 
		\hline
		Understand how to use hdf files & Computation of the loss & Preparation of the dataset & Finishing inference by creating segmentation \\
		\hline
		Create architecture of convolutional neural network & Find interesting patches in dataset & Get vertices pairs in same and different object & Learn PyTorch\\
		\hline
		Load ISBI-2012 data & Train and evaluate on ISBI in 2D & Train the network on maximin affinty & \\
		\hline
		Learn PyTorch & Learn PyTorch & Image reconstruction with inference & \\
		\hline
		  & & Fiji for evaluation on ISBI & \\
		  \hline
		  & & Load ISBI-2012 data & \\
		  \hline
		  & & Train and evaluate on ISBI in 2D & \\
		  \hline
		  & & Learn PyTorch & \\
		  \hline
	\end{tabular}
	\caption{List of the main tasks and their repartition (from our Trello board)}
	\label{tab:tasks}
\end{table}

As we can see in table~\ref{tab:tasks} we tried to distribute the work based on
affinity, and even though the table is unbalanced, everybody put their fair
share of work into the project.\\

One aspect that is absent from this table is all that we learned.
As we will detail in the next section, everybody didn't have the same knowledge
at the beginning of the project and it took everybody a different amount of
time to learn the necessary information.\\

\clearpage
\subsection{Obstacles and overcoming them}

The subject of this project was quite difficult at first because we did not have
any knowledge on image segmentation or morphology. Moreover, it was also hard to read and
understand the scientific papers because they were written in English and
we didn't have the knowledge to understand them clearly at first. To overcome this difficulty, we planned several
hours to go through the important things that we needed to retain, and over
time we were able to grasp more detail from the papers since our knowledge
improved throughout the project.\\

After knowing what we had do, the next step was to figure out how to implement
these ideas.\\
We took some time to know exactly what we needed to do and how to do it in detail for the implementation.
Once we knew, we just needed to find the right functions in Higra.\\
It took us some time to understand how the functions work, as we never used it before.
Documentation is available online and we spent a lot of time learning how to use the
functions.\\
Afterwards, we choose to use the PyTorch library for the deep learning
component of the project, which nobody was familiar with in the group. 
This lack of experience with it meant that we had to spend more time learning
it. However there is a nice tutorial on PyTorch's website that allowed us to
be able to use it fairly quickly. Their documentation on Autograd was also
really helpful when debugging the gradient history loss. Overall the PyTorch
API was really helpful and allowed us to use PyTorch with a relative ease.\\

During this project, we also received a lot of help from Quentin. This project would
have been really hard to do if he was not there, having no great knowledge in
image segmentation and machine learning. Thanks to him, we were able to have a
better understanding of the papers by having him explain them to us. It would have taken us
a lot longer to understand the papers and how to implement them without him and we would 
not be able to present decent results in the end of the first semester presentation.\\

Overall, being aware of the work done by others helped us immensely when
facing issues as we could all think together and challenge each other's
understanding of different topics. This allowed us to overcome all those
difficulties more easily and in a more enjoyable way.
